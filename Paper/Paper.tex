% Template taken from https://www.sharelatex.com/templates/journals/template-for-the-journal-of-machine-learning-research-jmlr

\documentclass[twoside,11pt]{article}

% Any additional packages needed should be included after jmlr2e.
% Note that jmlr2e.sty includes epsfig, amssymb, natbib and graphicx,
% and defines many common macros, such as 'proof' and 'example'.
%
% It also sets the bibliographystyle to plainnat; for more information on
% natbib citation styles, see the natbib documentation, a copy of which
% is archived at http://www.jmlr.org/format/natbib.pdf

\usepackage{jmlr2e}
\usepackage{amsmath}
% Definitions of handy macros can go here

\newcommand{\dataset}{{\cal D}}
\newcommand{\fracpartial}[2]{\frac{\partial #1}{\partial  #2}}

% Heading arguments are {volume}{year}{pages}{submitted}{published}{author-full-names}

\jmlrheading{1}{2019}{1-10}{9/19}{9/19}{George Engel, Troy Oster, Dana Parker, Henry Soule}

% Short headings should be running head and authors last names

\ShortHeadings{CSCI 447: Project 1}{Engel, Oster, Parker, Soule}
\firstpageno{1}

\begin{document}

\title{CSCI 447: Project 1}

\author{\name George Engel \email GeoEngel.z@gmail.com \\
       \addr Department of Engineering\\
       Montana State University\\
       Bozemane, MT 59715, USA
       \AND
       \name Troy Oster \email toster1011@gmail.com \\
       \addr Department of Engineering\\
       Montana State University\\
       Bozeman, MT 59715, USA
       \AND
       \name Dana Parker \email danaharmonparker@gmail.com \\
       \addr Department of Engineering\\
       Montana State University\\
       Bozeman, MT 59715, USA
       \AND
       \name Henry Soule \email hsoule427@gmail.com \\
       \addr Department of Engineering\\
       Montana State University\\
       Bozeman, MT 59715, USA}

\editor{Engel et al.}

\maketitle

\begin{abstract}%   <- trailing '%' for backward compatibility of .sty file

-Give general idea of what we are doing
-Summarize our results/findings here too

\end{abstract}

\begin{keywords}
    TODO: Enter keywords for assignment.
    Neural Networks, Machine Learning
%   
\end{keywords}

\section{Introduction}


\section{Problem Statement}
TODO: State what we want to do/how we plan to do it



\section{Hypotheses}

TODO: Add a general hypothesis, and then go into detail for each dataset.

\subsection{Abalone}

\subsection{Car}

\subsection{forestfires}

\subsection{machine}

\subsection{segmentation}

\section{The Algorithms}
TODO: Add description of the algorithm used, proper credits to an academic article cited in line, and how we went implemented this algorithm.

\section{Our Approach}
\subsection{Example}
\subsection{Example}
\subsection{Example}

% \subsection{Shuffling/Applying Noise}

% In addition to implementing our prediction algorithm for the plain version of each data set, we also ran it with a 10\% noise modifier applied to the training data set in order to try and grasp how noise might affect the naive bayes machine learning algorithm. 

% We implemented our prediction algorithm on the original version of each data set, and also introduced noise to each data set by randomly selecting 10\% of each data set and the shuffling attribute values. We then ran our algorithm on these shuffled versions of each data set.

\section{Results}
General summary of results goes here

\section{Conclusions}

TODO: Add general conclusion here

\subsection{Abalone}

\subsection{Car}

\subsection{Forest fires}

\subsection{Machine}

\subsection{Segmentation}


% Acknowledgements should go at the end, before appendices and references
\acks{}

% References go here (Don't forget inline comments!)

\end{document}